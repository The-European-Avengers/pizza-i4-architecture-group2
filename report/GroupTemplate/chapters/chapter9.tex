\section{Conclusion}

This section summarizes the main contributions of this work and outlines directions for future research.

\subsection{Summary of Contributions} 
This paper presented the design, verification, and implementation of a Pizza Production Experiment, a testbed for validating Industry 4.0 architectures. Our key contributions are:

\begin{itemize} 
    \item \textit{A Polyglot Event-Driven Architecture (RQ1)}: We successfully demonstrated how heterogeneous services (C\#, Python, Go) can interoperate in a manufacturing context using Kafka as a unified message bus.
    \item \textit{Formal Verification of Industrial Logic (RQ2)}: Using UPPAAL, we mathematically proved that the proposed distributed logic is free of deadlocks and satisfies safety constraints before writing a single line of code.
    \item \textit{Real-Time Observability (RQ3)}: We implemented a monitoring layer using ksqlDB that allows for the precise measurement of production KPIs (latency, restocking time) in real-time, providing transparency often missing in legacy systems.
\end{itemize}

\subsection{Future Work}

Future iterations of this work will focus on addressing the identified bottlenecks, expanding the system's capabilities, and addressing the remaining quality attributes:

\begin{itemize} 
    \item \textit{Horizontal Scaling}: Implementing a load balancer to distribute tasks among multiple instances of the Internal Goods Provider and production machines to eliminate the restocking queue bottleneck.
    \item \textit{Fault Tolerance Testing}: Introducing "Chaos Engineering" (e.g., randomly killing containers) to empirically validate the self-healing capabilities of the Docker restart policies and the resilience of the Kafka consumer groups.
\end{itemize}

