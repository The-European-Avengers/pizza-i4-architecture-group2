\section{Related Work}

As manufacturing systems evolve toward Industry 4.0, researchers increasingly focus on architectures supporting real-time data, flexible integration, and robustness under changing production demands\cite{kremer2023event}. Modern shop-floor modernization centers on two complementary approaches: exposing device data as real-time events and making application logic modular and independently deployable.

The Scania case demonstrates event-driven integration: PF6000 power tools stream tightening events as JSON to a Kafka plant bus, enabling real-time consumption by visualization, KPI analytics and maintenance systems, improving traceability while decoupling producers from consumers\cite{umer2018smart}. The MOAI proposal provides the complementary architectural view: modeling DAQ, M2M adapters, PLC/PID logic, and historians as microservices with a Transporter, API gateway, and security primitives (Guard/JWT) to enable discovery, load balancing, and containerized deployment effectively flattening the ISA-95 stack\cite{pontarolli2023microservice}.

Both patterns, event streaming and microservice automation, are emerging as dominant in Industry 4.0\cite{kremer2023event}, yet differ in emphasis: Scania provides concrete device integration for physical production lines, while MOAI offers architectural guidance for re-architecting automation stacks. Our work complements these by delivering a reproducible, containerized testbed with polyglot microservices, Kafka/ksqlDB streaming, and formal verification (UPPAAL). Deployed via Docker, it enables controlled experiments measuring latency, throughput and bottlenecks to evaluate scaling strategies before production deployment. This bridges device-level proofs-of-concept and architectural guidance: where Scania demonstrates shop-floor data streaming and MOAI shows service decomposition, our experiment validates their combined behavior under load, supporting evidence-based Industry 4.0 decisions.