\section{Introduction \& Motivation}

The adoption of Industry 4.0 requires restructuring traditional industrial systems from rigid hierarchies to flexible data-driven ecosystems\cite{kremer2023event}. Conventional automation architectures, such as the ISA-95 pyramid, often violate essential requirements for systemic flexibility and data availability\cite{kremer2023event}. To overcome these limitations, recent literature proposes Event-Driven Architectures (EDA) and Microservice-Oriented Architectures (MOA), which ensure component decoupling and data democratization\cite{pontarolli2023microservice}. 

EDA aligns with Industry 4.0 requirements by separating connectivity from functionality, enabling a centralized data space accessible to all components\cite{kremer2023event}. Practical implementations have emerged: Umer et al. demonstrated Apache Kafka as a \textit{Plant Service Bus} for collecting real-time data from smart power tools at Scania\cite{umer2018smart}, while Pontarolli et al. showed that microservices can flatten automation architectures through independent, network-based process control services\cite{pontarolli2023microservice}.

To validate these architectural concepts in a practical setting that mimics the complexity of heterogeneous manufacturing systems, this paper introduces the \textit{Pizza Production Experiment}, a fully containerized simulation of a distributed manufacturing line powered by Docker and Kafka. The primary motivation is developing a flexible architecture capable of evolving with changing production requirements, prioritizing the key quality attribute \textit{Performance}.

The structure of this paper is as follows. Section III reviews related work regarding Event-Driven and Microservice Architectures and their application in industrial settings. Section IV review the requirements and the domain analysis of the project. Section V details the proposed system architecture. Section VI describes the behavioral modeling and verification approach. Section VII details the experimental design, execution, and analysis. Section VIII presents an evaluation and discussion of the results, and Section IX concludes the work with key achievements and future work.